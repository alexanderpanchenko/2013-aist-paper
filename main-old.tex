\documentclass[a4paper,10pt,twoside]{article}
\usepackage{pattern-intuit}

\begin{document}

\article{Серелекс: поиск и визуализация семантически связанных слов}{Серелекс: поиск и визуализация семантически связанных слов}

\author{Панченко А.И.${}^{1,2}$, Романов П.В.${}^2$, Романов А.В${}^1$, \linebreak Филиппович А.Ю.${}^2$, Филиппович Ю.Н.${}^2$, Морозова О.Д.${}^1$}

\organization{${}^1$ Universit\'{e} catholique de Louvain, Лувен, Бельгия\linebreak 
${}^2$ МГТУ им. Н.\,Э. Баумана, Москва, Россия}

\annotation{Данная статья представляет Серелекс, систему, которая по запросу на английском языке предоставляет список семантически сязанных слов. Слова ранжируются в соответствии с оригинальной метрикой семантической близости, обученной на  корпусе естественно-языковых текстов. Точность работы системы сравнима с аналогами основанными на WordNet и других словарях. При этом наша система использует только информацию извлеченную непосредственно из текстов. Наше исследование показывает, что пользователи полностью удовлетворены результатами поиска семантически связанных слов в 70\% случаев.}{метрика семантической близости; визуализация семантических отношений}


\subsection{Введение}

Мы представляем Серелекс, систему, которая по заданному английскому слову, возвращает список связянных слов, отсортированных согласно их семантической близости. Система помогает изучить значение иностранных слов и интерактивно исследовать связанные лексические единицы. В отличие от аналогичных систем основанных на словарях и тезаурусах, таких как \url{Thesaurus.com} или \url{VisualSynonyms.com}, Серелекс полагается на информацию, извлечённую из корпусов текстов. По сравнению с другими подобными системами, такими как BabelNet~\footnote{ \url{http://lcl.uniroma1.it/bnxplorer/}}, ConceptNet~\footnote{ \url{http://conceptnet5.media.mit.edu/}} или UBY~\footnote{\url{https://uby.ukp.informatik.tu-darmstadt.de/webui/tryuby/}}, Серелекс не зависит от семантических ресурсов, таких как WordNet. Система использует оригинальную метрику семантической близости между словами основанную на лексико-синтаксических шаблонах~\cite{panchenko2012konvens}. Согласно нашим экспериментам, данный подход имеет точность сопоставимую с 9 альтернативными метриками предложенными в предыдущих исследованиях. Кроме тогои, имеет большее лексическое покрытие, чем системы, основанные на словаре, обеспечивает интерфейс 
пользовтаеля в виде списка, графа и основанного на изображениях, и является системой с открытым исходным кодом.

\subsection{Система}

Серелекс находится в открытом доступе в Интернет~\footnote{\url{http://serelex.cental.be} или \url{http://serelex.it-claim.ru}}.
На рис. \ref{fig-arch2} изображена архитектура системы, которая состоит из экстрактора, сервера и пользовательсокого интерфейса. Экстрактор извлекает семантические отношения между словами из необработанного корпуса текстов. Извлечённые отношения сохраняются в базе данных. Сервер обеспечивает быстрый доступ к извлечённым отношениями по HTTP. Пользователь взаимодействует с системой через веб-интерфейс или API. Система, а так же данные и скрипты оценки имеют открытый исходный код~\footnote{ \url{http://serelex.cental.be/page/about}}, доступный под лицензией LGPLv3.

\risunok{Архитектура системы.}{arch2}{8cm}

\textbf{Система извлечения семантических отношений.} Подсистема извлечения основана на семантической мере подобия \textit{PatternSim} и  формуле ранжирования \textit{Efreq-Rnum-Cfreq-Pnum}~\cite{panchenko2012konvens}. Эта, основанная на корпусе мера, полагается на лексико-синтактические шаблоны, которые извлекают конкордансы. Сходство пропорционально числу со-возникновений термина в конкордансах, например: \texttt{  such \{non-alcoholic [sodas]\} as \{[root beer]\} and \{[cream soda]\}}. Оценка нормализуется с частотой термина и другими извлечёнными статистиками~\cite{panchenko2012konvens}. В качестве корпуса мы использовали комбинацию кратких обзоров Википедии и ukWaC~\cite{baroni2009wacky} (5 387 431 документов, 2.915 $\cdot 10^9$ тоекнов, 7,585,989 лемм, 17.64 Гб). Обработка корпуса заняла около 70 часов на обычной машине (Intel i5, 4Гб ОЗУ, HDD 5400 об/мин). В результате выделено  11 251 240 нетипизированных семантических отношений (например, $\langle Canon, Nikon, 0.62 \rangle$) между 419,751 словами. 

\textbf{Сервер.} Сервер возвращает список связанных слов для каждого запроса, отсортированных согласно их семантическому сходству, сохранённому в базе данных. Запросы лемматизируются при помощи словаря DELA~\footnote{\url{http://infolingu.univ-mlv.fr/}, доступно под лицензией LGPLLR.}. Для запросов без результатов выполняется приблизительный поиск. Система может импортировать сети в формате CSV, созданные другими метриками сходства и экстракторами.

\textbf{Пользовательский интерфейс.} К системе можно получить доступ через веб-интерфейс или RESTfull API. Графичесикй интерфейс пользователя состоит из трёх основных элементов: поле поиска, список с результатами и граф результатов (см рис. \ref{fig-gui}). Пользователь взаимодействует с системой, вводя запрос -- одно слово, например ``mathematics'', или несколько слов, например ``computational linguistics''. 

\risunok{Графический интерфейс пользователя.}{jaguar}{11cm}

Кроме графового интерфейса пользователя, реализован интерфейс, основанный на изображениях. При это, всю рабочую область занимают графическое представление слов, связанных с данным. Изображения получаются с помощью сервиса jpg.to~\footnote{\url{http://jpg.to/about.php}, jpg.to использует Google Image Search \url{http://images.google.ru/}}. По клику на изображение происходит переход к словам, семантически связанным с нажатым.

\risunok{Интерфейс, основанный на изображениях.}{images}{10cm}

Так же, были разработаны приложения для Windows 8~\footnote{\url{http://apps.microsoft.com/windows/app/lsse/48dc239a-e116-4234-87fd-ac90f030d72c}} и Windows Phone~\footnote{\url{http://www.windowsphone.com/s?appid=dbc7d458-a3da-42bf-8da1-de49915e0318}}. Данные приложения используют RESTfull API для получения результатов запроса пользвателя и также используют сервис jpg.to для получения изображений и выполненны с учетом рекомендаций по построеноию польховательского интерфейса приложений для Windows и Windows Phone. В рамках создания приложений для Windows 8 была создана переносимая бибилиотека классов (Portable Class Library), а, так как исходные коды приложения для Windows 8 доступны на Github~\footnote{\url{https://github.com/jgc128/Serelex4Win}}, это предоставляет возможность сторонним программистам создавать свои приложений на платформе .NET 4.5, использующие сервис Серелекс, без написание своего кода для доступа к RESTfull API. Отличительной особеностью приложения для Windows Phone является то, что оно 
позволяет сразу же выполнить поиск в Google по результатам запроса благодаря наличию специальной кнопки.

\risunok{Скриншот приложения для Windows 8.}{win8}{10cm}

\risunok{Скриншот приложения для Windows Phone.}{wp}{10cm}

\subsection{Результаты}
Мы оценивали систему, исходя из четырех задач (см. ~\cite{panchenko2012konvens} для подробностей):

\subsubsection{Корреляция с человеческим суждением} Мы использовали стандартные наборы данных для измерения корреляции Спирмена с человеческим суждением. Наша система сравнивается с базовыми результатами, включающими 3 метрики, основанные на WordNet (\textit{WuPalmer}~\cite{wu1994verbs}, \textit{LeacockChodorow}~\cite{leacock1998}, \textit{Resnik}~\cite{resnik1995}), 3 основанные на словарях (\textit{ExtendedLesk}~\cite{banerjee2003extended}, \textit{GlossVectors}~\cite{patwardhan2006using}, \textit{WiktionaryOverlap}~\cite{zesch2008extracting}), и 3 метрики, основанные на корпусах (\textit{ContextWindow}~\cite{cruys2010mining}, \textit{SyntacticContext}~\cite{cruys2010mining}, ~\textit{LSA}~\cite{landauer1998introduction}).

\subsubsection{Ранжирование семантических отношений} Эта задача опирается на набор семантических отношений (BLESS,  SN) для оценивания \textit{относительной} точности и полноты каждой метрики. Точность \textit{Серелекс}  сопоставима с 9 базовыми метриками, но её полнота серьезно ниже, в связи с разреженностью подхода, основанного на шаблонах (см. Рис \ref{fig-eval} (a)).

\subsubsection{Извлечение семантических отношений} Мы оценивали точность извлеченных отношений для 49 слов (словарь надора данных RG). Три аннотатора указывали, связанныли термины семантическими отношениями, или нет. Каждому из них было предложено отметить, релевантны ли первые 50 результатов, или нет. Мы вычислили точность извлечения как $k = \{1, 5, 10, 20, 50\}$. Средняя точность варьируется между 0,736 для первых результатов, и 0,599 для первых 50 результатов (см. Рис \ref{fig-eval} (b)). Степень согласия в терминах каппы Флейса значитальеная (0.61-0.80).

\risunok{Оценка: (a) Граф точность-полнота задачи ранжирования семантических отношений в BLESS; (b) задача извлечения семантических отношений; (c) удовлетворенеие пользователей от первых 20 резальтатах}{eval}{11cm}

\subsubsection{Удовлетворение пользователей} Мы также имзмеряли удовлетворение пользователей от наших результатах. 23 экспертам было предложенно выбрать 20 запросов по своему усмотрению и ранжировать первые 20 результатов как релевантные, нерелевантыне и частично редевантные для каждого из запросов. Мы собрали 460 решений экспертов и 233 решения анонимных пользователей (см. Рис \ref{fig-eval} (с)). Пользователи и эксперты (пользоваталей просили воспольховаться системой) вместе создали 594 уникальных запросов. В соответствии с этим экспериментов, результаты были редлевантын  в 70\% случаях, и нерелевантны в 10\% случаях. Наконец, в 20\% запросов были релевантные и нерелевантные результаты.

\subsection{Выводы}

Мы представиди систему, котроая находит семантически связанные слов. Наши результаты показывают точность, вопоставимую с подходом, основанным на словарях и обладают лучшим покрытием, так как отношения извлекаются непосредственно из текста. Система достигает точности@1 в около 74\%, и удовлетворения пользователей в 70\% результатов запросов без необхожимости какго-либо ручного составления словаря.


\bibliographystyle{splncs}
\bibliography{biblio2}

\end{document}
