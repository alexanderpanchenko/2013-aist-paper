%%%%%%%%%%%%%%%%%%%%%%%%%%%%%%%%%%%%%%%%%%%%%%%%%
\begin{frame}
\frametitle{Статьи и тезисы опубликованные по теме диссертации}


\begin{enumerate}
  \footnotesize
\item Panchenko A. Morozova O. “A Study of Hybrid Similarity Measures for Semantic Relation Extraction”. Submitted to Innovative Hybrid Approaches to the Processing of Textual Data Workshop of EACL, 2012
\item Panchenko A. “A Large-Scale Study of Similarity Measures for Semantic Relation Extraction”. Submitted to TALN 2012,  2012

\item Панченко A., Адейкин С., Романов П., Романов A., “Извлечение семантических отношений из статей Википедии с помощью алгоритмов ближайших соседей”. Анализ Социальных сетей, Изображений и Текстов (АИСТ), Екатеринбург, 2012

\item Panchenko A. “Towards an Efficient Combination of Similarity Measures for Semantic Relation Extraction” (Abstract) In Computational Linguistics in the Netherlands CLIN 22,  2012

\item Panchenko A. “Comparison of the Knowledge-, Corpus-, and Web-based Similarity Measures for Semantic Relations Extraction”. In Proceedings of GEometrical Models of Natural Language Semantics, EMNLP, 2011 

\item Панченко A. Метод автоматического построения семантических отношений между концептами информационно-поискового тезауруса. Вестник ВГУ, Серия: Системный анализ и информационные технологии, 2010, No 2, стр.131–139, 2011. 
 
\end{enumerate}

\end{frame}

\begin{frame}
\frametitle{Статьи и тезисы опубликованные по теме диссертации}
\begin{enumerate}
  \footnotesize
  \setcounter{enumi}{6}
  
  
  
\item Panchenko A. “Computing Semantic Relations from Heterogeneous Evidence”.  (Abstract) In Computational Linguistics in the Netherlands CLIN 21, Ghent, pp.39, 2011.

\item Панченко А. "Технология построения информационно-поискового тезауруса", Интеллектуальные Системы и Технологии, Эликс+, том. 9, стр.124-140, 2009. 

\item Панченко А. "Построение тезауруса из корпуса текстов предметной области". (Тезисы) Конф.Жизнь языка в культуре и обществе, Институт языкознания РАН, 2009 

\item Панченко А. “Методы автоматического извлечения семантических отношений” (Тезисы). Конф.Электронные средства информации в современном обществе, МГУП, 2010  

\item Панченко А. “Автоматизированная система построения тезауруса”. Дипломная работа. МГТУ им.Н.Э.Баумана, 177 страниц, 2008 

\item Панченко А. “Средство анализа ассоциативных сетей”, Интеллектуальные Системы и Технологии, том.9, стр.169-203, Эликс+, 2008

\end{enumerate}



\end{frame}

