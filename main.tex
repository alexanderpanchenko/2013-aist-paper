\documentclass[a4paper,10pt,twoside]{article}
\usepackage{pattern-intuit}

\begin{document}

\article{Серелекс: поиск и визуализация семантически связанных слов}{Серелекс: поиск и визуализация семантически связанных слов}

\author{Панченко А.И.${}^{1,2}$, Романов П.В.${}^2$, Романов А.В${}^1$, \linebreak Филиппович А.Ю.${}^2$, Филиппович Ю.Н.${}^2$, Морозова О.И.${}^1$}

\organization{${}^1$ Universit\'{e} catholique de Louvain, Лувен, Бельгия\linebreak 
${}^2$ МГТУ им. Н.\,Э. Баумана, Москва, Россия}

\annotation{В статье представлена семантическая система Серелекс, которая выдает в ответ на поисковый запрос список семантически связанных с ним слов. В настоящее время система работает только на английском языке, ведутся также разработки для французского и русского языков. Слова ранжируются в соответствии с оригинальной метрикой семантической близости, обученной на  корпусе естественно-языковых текстов. Точность работы системы сравнима с аналогами, основанными на WordNet и словарях. При этом система использует только информацию, извлеченную непосредственно из текстов. Исследование показывает, что пользователи полностью удовлетворены результатами поиска семантически связанных слов в 70\% случаев.}{метрика семантической близости; визуализация семантических отношений}

\subsection{Введение}

В данной статье представлена система Серелекс, которая выдает в ответ на английский запрос список связанных с ним слов в порядке их семантической близости~\footnote{Данная статья является расширенной версией~\cite{panchenko2013serelex}.}. Программа помогает изучить значение иностранных слов и интерактивно исследовать связанные лексические единицы и их семантические поля. В отличие от аналогичных систем, основанных на словарях и тезаурусах, таких как \url{Thesaurus.com} или \url{VisualSynonyms.com}, Серелекс использует информацию, извлечённую из корпуса естественно-языковых текстов и не использует информацию из семантических ресурсов, таких как WordNet, в отличие от других подобных систем (BabelNet~\footnote{ \url{http://lcl.uniroma1.it/bnxplorer/}}, ConceptNet~\footnote{ \url{http://conceptnet5.media.mit.edu/}} или UBY~\footnote{\url{https://uby.ukp.informatik.tu-darmstadt.de/webui/tryuby/}}). В основе системы лежит оригинальная метрика семантической близости, использующая лексико-синтаксических шаблоны~\cite{panchenko2012konvens}. Согласно экспериментам, точность использованного подхода сопоставима с существующими метриками. Кроме этого, представляемая система характеризуется большим лексическим покрытием, чем аналоги, основанные на словарах, предлагает три альтернативных способа визуализации результатов запроса (в виде списка, графа и набора изображений) и имеет открытый исходный код.

\subsection{Система}

Серелекс находится в открытом доступе в интернете~\footnote{\url{http://serelex.cental.be} или \url{http://serelex.it-claim.ru} }.
Система состоит из экстрактора, сервера и пользовательского интерфейса (см. Рис.~\ref{fig-drawing}). Задача экстрактора заключается в извлечении семантических отношений между словами из корпуса естестенно-языковых текстов на английском языке. Извлечённые отношения сохраняются в базе данных. Сервер обеспечивает быстрый доступ к извлечённым отношениями через HTTP. Пользователь взаимодействует с системой через веб-интерфейс или API. Исходный код системы, данные и скрипты оценки качества работы доступны на условиях лицензии LGPLv3~\footnote{ \url{http://serelex.cental.be/page/about} }.

\risunok{Архитектура системы.}{drawing}{10cm}

\textbf{Экстрактор.} Подсистема извлечения семантических отношений основана на метрике семантической близости \textit{PatternSim} и формуле ранжирования \textit{Efreq-Rnum-Cfreq-Pnum}~\cite{panchenko2012konvens}. Метрика семантической близости использует лексико-синтаксические шаблоны, подобно~\cite{hearst1992}. Данные шаблоны извлекают из корпуса текстов множество конкордансов, таких как: 

\begin{itemize}
\footnotesize
\item \texttt{such diverse \{[occupations]\} as \{[doctors]\}, \{[engineers]\} and \{[scientists]\}}

\item \texttt{such \{non-alcoholic [sodas]\} as \{[root beer]\} and \{[cream 
 soda]\}}
 
\item \texttt{\{traditional[food]\}, such as \{[sandwich]\}, \{[burger]\}, and \{[fries]\}}

\item \texttt{\{[mango]\},\{[pineapple]\}, \{[jackfruit]\} and other\{[fruits]\}}

\item \texttt{\{primitive [snake]\}, such as \{[boa]\} and \{[python]\}}
\item \texttt{\{[France]\},\{[ Belgium]\} and other \{European [countries]\}}
\end{itemize}


Слова в конкордансах были лемматизированы с помощью словаря DELA~\footnote{\url{http://infolingu.univ-mlv.fr/}, доступен на условиях лицензии LGPLLR}. Семантическое сходство двух лемм пропорционально количеству конкордансов, в которых они совместно встретились. Однако окончательное значение семантической близости вычисляется с учетом и других факторов, таких как частота слов в корпусе и количество извлеченных отношений для каждого из слов~\cite{panchenko2012konvens}. Было произведено извлечение отношений из коллекции текстовых документов, состоящей из заголовков статей Википедии и корпуса ukWaC~\cite{baroni2009wacky} (см. Таблицу~\ref{tbl:corpora}). Обработка данного корпуса заняла около 72 часов на стандартном компьютере (Intel i5, 4Гб ОЗУ, HDD 5400 об/мин). В результате извлечения было выявлено 11,251,240 нетипизированных семантических отношений, таких как  $\langle Canon, Nikon, 0.62 \rangle$, между 419,751 леммами. 

\begin{table}
\centering
\footnotesize
\begin{tabular}{|l|c|c|c|c|}
  \hline              
  Название & \# Документов & \# Словоформ & \# Лемм &  Размер \\ \hline         \hline           
  Википедия & 2,694,815 & 2,026 $\cdot$ 10^9 & 3,368,147 & 5.88 Гб \\
  ukWaC & 2,694,643 & 0.889 $\cdot$ 10^9 & 5,469,313 & 11.76 Гб \\ 
  Википедия + ukWaC & 5,387,431 & 2.915 $\cdot$ 10^9 & 7,585,989 & 17.64 Гб\\
  \hline  
\end{tabular}
\caption{Корпуса текстов, использованные системой.}
\label{tbl:corpora}
\end{table}

\textbf{Сервер.} Сервер возвращает множество связанных слов для каждого запроса, отсортированных согласно их семантической близости, сохранённой в базе данных. Запросы перед обработкой лемматизируются при помощи словаря DELA. Для слов, на которые не нашлось ни одного результата, выполняется приблизительный поиск с помощью расстояния Левенштейна. Система позволяет импортировать семантические отношения в формате CSV, которые были извлечены альтернативными экстракторами.

\textbf{Пользовательский интерфейс.} Для работы с системой можно использовать доступ через веб-интерфейс, приложения для платформы Windows 8, Windows Phone либо RESTful веб-сервис. Веб-интерфейс состоит из трёх основных элементов: строки поиска, списка результатов и графа результатов (см. Рис.~\ref{fig-jaguar}). Пользователь взаимодействует с системой, формулируя поисковый запрос, который может быть выражен словом, таким как ``mathematics'', или словосочетанием, таким как ``computational linguistics''. 

\risunok{Визуализация результатов поиска.}{jaguar}{11cm}

Кроме графового интерфейса пользователя, реализован интерфейс, основанный на изображениях. При этом всю рабочую область занимают графическое представление слов, связанных с результатами поиска. Выбор изображений осуществляется на основе веб-сервиса jpg.to~\footnote{\url{http://jpg.to/about.php}. Данный сервис использует Google Image Search: \url{http://images.google.ru/}.}. Кликнув на изображение, пользователь может перейти к словам, семантически связанным с словом, представленном на изображении.

\risunok{Интерфейс основанный на изображениях.}{images}{11cm}

В дополнение к веб-интерфейсу были разработаны приложения для Windows 8~\footnote{\url{http://apps.microsoft.com/windows/app/lsse/48dc239a-e116-4234-87fd-ac90f030d72c}} и Windows Phone~\footnote{\url{http://www.windowsphone.com/s?appid=dbc7d458-a3da-42bf-8da1-de49915e0318}}. Данные клиенты используют веб-сервис Серелекса для получения результатов запросов и сервис jpg.to для получения изображений (см. Рис.~\ref{fig-drawing}). Приложения выполнены с учетом рекомендаций по построению пользовательского интерфейса приложений для Windows и Windows Phone, а их исходный код является открытым~\footnote{\url{https://github.com/jgc128/Serelex4Win}}. В рамках создания приложений для Windows была создана переносимая билиотека классов (Portable Class Library), которая может быть полезна для доступа к веб-сервису Серелекса из сторонних приложений. Отличительной особенностью клиента системы для Windows Phone является то, что он позволяет сразу же выполнить поиск в Google по результатам запроса (см. Рис.~\ref{fig-wp}).

\risunok{Клиент системы для платформы Windows 8.}{win8}{10cm}

\risunok{Клиент системы для Windows Phone 8.}{wp}{10cm}

\subsection{Результаты}
Оценим качество работы системы, проведя четыре эксперимента, подробное описание которых приведено в~\cite{panchenko2012konvens}.  

\subsubsection{Корреляция с суждениеми о семантической близости} Для оценки корреляции с суждениеми о семантической близости использовались три проверочных набора данных широко, распространенных в англоязычной литературе по лексической семантике: \textit{MC}~\cite{miller1993semantic}, \textit{RG}~\cite{rubenstein1965} и \textit{WordSim}~\cite{finkelstein2001placing}. Данные коллекции содержат множество пар слов, для каждой из которых вручную задана мера их семантической близости:
\begin{itemize}
  \footnotesize
  \item \texttt{automobile; car; 3.92}
  \item \texttt{brother; monk; 2.84}
  \item \texttt{glass; magician; 0.11}
\end{itemize}

 Согласно результатам проведенных экспериментов корреляция Спирмена между значениями семантической близости, предоставляемыми системой, и суждениями субъектов достигает 
0.665, 0.739 и 0.520 для \textit{MC}, \textit{RG} и \textit{WordSim} соответственно. Данные характеристики Серелекс сравнимы с показателями существующих метрик семантической близости, основанных на WordNet (\textit{WuPalmer}~\cite{wu1994verbs}, \textit{LeacockChodorow}~\cite{leacock1998}, \textit{Resnik}~\cite{resnik1995}), словарях (\textit{ExtendedLesk}~\cite{banerjee2003extended}, \textit{GlossVectors}~\cite{patwardhan2006using}, \textit{WiktionaryOverlap}~\cite{zesch2008extracting}) и  корпусах текстов (\textit{ContextWindow}~\cite{cruys2010mining}, \textit{SyntacticContext}~\cite{cruys2010mining}, ~\textit{LSA}~\cite{landauer1998introduction}).

\subsubsection{Ранжирование семантических отношений}

В данном тесте нужно отсортировать некоторое множество слов по семантической близости с заданным словом. Например, дано 50 слов, 25 из которых связано со словом ``alligator'', а 25 других с ним не связано. Задача системы в ранжировании слов таким образом, чтобы семантически связанные термины имели более высокий ранг:

\begin{itemize}
\footnotesize
\item \texttt{1; alligator; animal (related)}
%\item \texttt{2; alligator; beast (related)}
\item \ldots
\item \texttt{25; alligator; lizard (related)}
\item \texttt{26; alligator; twin (random)}
\item \ldots
\item \texttt{50; alligator; electronic (random) }
\end{itemize}

Данный тест основан на наборе семантических отношений BLESS~\cite{baroniwe} и SN~\cite{panchenko2012study}. Точность Серелекса на данной задаче сопоставима с 9 названными выше альтернативными метриками, однако полнота серьезно ниже в связи с разреженностью подхода, основанного на шаблонах (см. Рис \ref{fig-eval} (a)).

\subsubsection{Извлечение семантических отношений} 

Кроме двух описанных выше тестов, была оценена точность извлечения семантических отношений 49 слов из лексикона \textit{RG}. В данном эксперименте трем асессорам было предложено аннотировать результаты поиска и указть для каждого из 50 первых результатов поиска, является ли он релевантным или нет. Например, для запроса ``fruit'':

\begin{itemize}
  \footnotesize
  \item \texttt{1; vegetable (relevant)}
  \item \texttt{2; mango (relevant)}
  \item \ldots
  \item \texttt{50; house (non-relevant)}
\end{itemize} 

На основании полученной статистики вычислена точность для $k$, где $k \in \{1, 5, 10, 20, 50\}$. Согласно результатам данного эксперимента, приведенным на Рис \ref{fig-eval} (б), средняя точность извлечения варьируется между 74\% (для первого результата, $k=1$) и 56\% (50 первых результатов, $k=50$). Зафиксирована значительная степень согласия асессоров для данного эксперимента в терминах каппы Флейса (0.61-0.80).

\risunok{Результаты: (a) задача ранжирования семантических отношений; (б) задача извлечения семантических отношений; (в) удовлетворенность пользователей первыми 20 результатами поиска.}{eval}{11cm}

\subsubsection{Удовлетворенность пользователей качеством поиска} Каждому из 23-х асессоров, участвующих в исследовании, было предложено выбрать 20 запросов по своему усмотрению и оценить первые 20 результатов поиска как релевантные, нерелевантные или как частично релевантные. В результате данной оценки собранно 460 суждений асессоров и 233 суждения анонимных пользователей системы. Пользователи и асессоры вместе осуществили 594 уникальных запроса. В соответствии с этим экспериментом, результаты поиска являются релевантными для 70\% запросов и нерелевантными для 10\% запросов (см. Рис \ref{fig-eval} (в)). В 20\% случаев первые 20 результатов оказались частично релевантными.

\subsection{Выводы}

Разработанна система Серелекс, которая позволяет осуществлять поиск семантически связанных слов. Оценка качества работы системы на четырех экспериментах показала, что точность системы сопоставима с аналогичными разработками, предложенными ранее. При этом в отличие от большинства аналогов Серелекс не использует составленные вручную словари. За счет этого достигается лучшее лексическое покрытие, т.к. семантические отношения извлекаются непосредственно из текста. Опрос пользователей показал, что первый результат поиска релевантен в 74\% случаях и в 70\% запросов пользователи полностью удовлетворены первыми 20 результатами.   

\bibliographystyle{splncs}
\bibliography{biblio2}

\end{document}
