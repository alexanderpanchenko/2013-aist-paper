\documentclass[a4paper,10pt,twoside]{article}
\usepackage{pattern-intuit}

\begin{document}

\article{Серелекс: поиск и визуализация семантически связанных слов}{Серелекс: поиск и визуализация семантически связанных слов}

\author{Панченко А.И.${}^{1,2}$, Романов П.В.${}^2$, Романов А.В${}^1$, \linebreak Филиппович А.Ю.${}^2$, Морозова О.И.${}^1$, Филиппович Ю.Н.${}^2$}

\organization{${}^1$ Universit\'{e} catholique de Louvain, Луван-ла-Нев, Бельгия\linebreak 
${}^2$ МГТУ им. Н.\,Э. Баумана, Москва, Россия}

\annotation{В статье представлена система Серелекс, которая выдает в ответ на поисковый запрос список семантически связанных с ним слов. В настоящее время система работает на английском языке, ведутся также разработки для французского и русского языков. Слова ранжируются в соответствии с оригинальной метрикой семантической близости, обученной на  корпусе естественно-языковых текстов. Точность работы системы сравнима с аналогами, основанными на WordNet и словарях. При этом система использует только информацию, извлеченную непосредственно из текстов. Исследование показывает, что пользователи полностью удовлетворены результатами поиска семантически связанных слов в 70\% случаев.}{метрика семантической близости; визуализация семантических отношений}

\subsection{Введение}

В данной статье представлена система Серелекс, которая на английский запрос выдает список связанных с ним слов в порядке их семантической близости~\footnote{Данная статья является расширенной версией~\cite{panchenko2013serelex}.}. Программа помогает изучить значение иностранных слов и интерактивно исследовать связанные лексические единицы и их семантические поля. В отличие от систем, основанных на словарях и тезаурусах, таких как \url{Thesaurus.com} или \url{VisualSynonyms.com}, Серелекс использует информацию, извлечённую из корпуса естественно-языковых текстов. В отличие от аналогичных систем, извлекающих информацию из текстов, таких как BabelNet~\footnote{ \url{http://lcl.uniroma1.it/bnxplorer/}}, ConceptNet~\footnote{ \url{http://conceptnet5.media.mit.edu/}} и UBY~\footnote{\url{https://uby.ukp.informatik.tu-darmstadt.de/webui/tryuby/}}, Серелекс не использует дополнительно информацию из таких семантических ресурсов, как WordNet. 

В основе разработанной системы лежит оригинальная метрика семантической близости, использующая лексико-синтаксические шаблоны~\cite{panchenko2012konvens}. Согласно экспериментам, точность использованного подхода сопоставима с существующими аналогами для английского языка. Кроме того, представленная система характеризуется большим лексическим покрытием, чем аналоги, основанные на словарях, предлагает три альтернативных способа визуализации результатов запроса (в виде списка, графа и набора изображений) и имеет открытый исходный код.

\subsection{Система}

Серелекс находится в открытом доступе в интернете~\footnote{\url{http://serelex.cental.be} или \url{http://serelex.it-claim.ru} }.
Система состоит из экстрактора, сервера и пользовательского интерфейса (см. Рис.~\ref{fig-drawing}). Задача экстрактора заключается в извлечении семантических отношений между словами из корпуса естественно-языковых текстов. Извлечённые отношения сохраняются в базе данных. Сервер обеспечивает быстрый доступ к извлечённым отношениями через HTTP. Пользователь взаимодействует с системой через веб-интерфейс или API. Исходный код системы, данные и скрипты оценки качества работы доступны на условиях лицензии LGPLv3~\footnote{ \url{http://serelex.cental.be/page/about} }.

\risunok{Архитектура системы.}{drawing}{10cm}

\subsubsection{Экстрактор} Подсистема извлечения семантических отношений основана на метрике семантической близости \textit{PatternSim} и формуле ранжирования \textit{Efreq-Rnum-Cfreq-Pnum}~\cite{panchenko2012konvens}. Метрика семантической близости использует лексико-синтаксические шаблоны, подобно~\cite{hearst1992}. Данные шаблоны извлекают из корпуса текстов множество конкордансов, таких как: 

\begin{itemize}
\footnotesize
\item \texttt{such diverse \{[occupations]\} as \{[doctors]\}, \{[engineers]\} and \{[scientists]\}}

\item \texttt{such \{non-alcoholic [sodas]\} as \{[root beer]\} and \{[cream 
 soda]\}}
 
\item \texttt{\{traditional[food]\}, such as \{[sandwich]\}, \{[burger]\}, and \{[fries]\}}

\item \texttt{\{[mango]\},\{[pineapple]\}, \{[jackfruit]\} and other\{[fruits]\}}

\item \texttt{\{primitive [snake]\}, such as \{[boa]\} and \{[python]\}}
\item \texttt{\{[France]\},\{[ Belgium]\} and other \{European [countries]\}}
\end{itemize}


Слова в конкордансах были лемматизированы с помощью словаря DELA~\footnote{\url{http://infolingu.univ-mlv.fr/}, доступен на условиях лицензии LGPLLR.}. Семантическое сходство двух лемм пропорционально количеству конкордансов, в которых они совместно встретились. Однако окончательное значение семантической близости вычисляется с учетом и других факторов, таких как частота слов в корпусе и количество извлеченных отношений для каждого из слов~\cite{panchenko2012konvens}. Извлечение отношений было произведено из коллекции текстовых документов, состоящей из заголовков статей Википедии и корпуса ukWaC~\cite{baroni2009wacky} (см. Таблицу~\ref{tbl:corpora}). Обработка данного корпуса заняла около 72 часов на стандартном компьютере (Intel i5, 4Гб ОЗУ, HDD 5400 об/мин). В результате извлечения было выявлено 11,251,240 нетипизированных семантических отношений, таких как  $\langle Canon, Nikon, 0.62 \rangle$, между 419,751 леммами. 


\begin{table}
\centering
\footnotesize
\begin{tabular}{|l|c|c|c|c|}
  \hline              
  Название & \# Документов & \# Словоформ & \# Лемм &  Размер \\ \hline         \hline           
  Википедия & 2,694,815 & 2,026 $\cdot$ 10^9 & 3,368,147 & 5.88 Гб \\
  ukWaC & 2,694,643 & 0.889 $\cdot$ 10^9 & 5,469,313 & 11.76 Гб \\ 
  Википедия + ukWaC & 5,387,431 & 2.915 $\cdot$ 10^9 & 7,585,989 & 17.64 Гб\\
  \hline  
\end{tabular}
\caption{Корпуса текстов, использованные системой.}
\label{tbl:corpora}
\end{table}

\subsubsection{Сервер} Сервер возвращает множество связанных слов для каждого запроса, отсортированных согласно их семантической близости, сохранённой в базе данных. Запросы перед обработкой лемматизируются при помощи словаря DELA. Для слов, для которых не нашлось ни одного результата, выполняется приблизительный поиск с помощью расстояния Левенштейна. Система позволяет импортировать семантические отношения, которые были извлечены альтернативными экстракторами, в формате CSV.

\subsubsection{Пользовательский интерфейс} Для работы с системой можно использовать веб-интерфейс, приложение для Windows 8, приложение для Windows Phone 8 или RESTful веб-сервис. Веб-интерфейс состоит из трёх основных элементов: строки поиска, списка результатов и графа результатов (см. Рис.~\ref{fig-jaguar}). Пользователь взаимодействует с системой, формулируя поисковый запрос, который может быть выражен словом, таким как ``mathematics'', или словосочетанием, таким как ``computational linguistics''. 

\risunok{Визуализация результатов поиска.}{jaguar}{11cm}

Кроме графового интерфейса пользователя, реализован интерфейс, основанный на изображениях. При этом всю рабочую область занимает графическое представление слов, связанных с результатами поиска. Выбор изображений осуществляется на основе веб-сервиса jpg.to~\footnote{\url{http://jpg.to/about.php}. Данный сервис использует Google Image Search: \url{http://images.google.ru/}.}. Кликнув на изображение, пользователь может перейти к словам, семантически связанным с словом, представленном на изображении.

\risunok{Интерфейс, основанный на изображениях.}{images}{10cm}
\risunok{Клиент системы для платформы Windows 8.}{win8}{10cm}
\risunok{Клиент системы для Windows Phone 8.}{wp}{10cm}

В дополнение к веб-интерфейсу были разработаны приложения для Windows 8~\footnote{\url{http://apps.microsoft.com/windows/app/lsse/48dc239a-e116-4234-87fd-ac90f030d72c}} и Windows Phone~\footnote{\url{http://www.windowsphone.com/s?appid=dbc7d458-a3da-42bf-8da1-de49915e0318}}. Данные клиенты используют веб-сервис Серелекса для получения результатов запросов и сервис jpg.to для получения изображений (см. Рис.~\ref{fig-drawing}). Приложения выполнены с учетом рекомендаций по построению пользовательского интерфейса приложений для Windows и Windows Phone, а их исходный код является открытым~\footnote{\url{https://github.com/jgc128/Serelex4Win}}. В рамках создания приложений для Windows была создана переносимая билиотека классов (Portable Class Library), которая может быть полезна для доступа к веб-сервису Серелекса из сторонних приложений. Отличительной особенностью клиента системы для Windows Phone является то, что он позволяет сразу же выполнить поиск в Google по результатам запроса (см. Рис.~\ref{fig-wp}).

\subsection{Результаты}
Мы оценили качество работы системы, проведя четыре эксперимента, подробное описание которых приведено в~\cite{panchenko2012konvens}.  

\subsubsection{Корреляция с суждениями о семантической близости} 
Для оценки корреляции с суждениями о семантической близости использовались три проверочных набора данных, широко распространенных в англоязычной литературе по лексической семантике: \textit{MC}~\cite{miller1993semantic}, \textit{RG}~\cite{rubenstein1965} и \textit{WordSim}~\cite{finkelstein2001placing}. Данные коллекции содержат множество пар слов, для каждой из которых вручную задана мера их семантической близости, например:
\begin{itemize}
  \footnotesize
  \item \texttt{automobile; car; 3.92}
  \item \texttt{brother; monk; 2.84}
  \item \texttt{glass; magician; 0.11}
\end{itemize}

 Согласно результатам проведенных экспериментов, корреляция Спирмена между значениями семантической близости, предоставляемыми системой, и суждениями субъектов достигает 
0.665, 0.739 и 0.520 для \textit{MC}, \textit{RG} и \textit{WordSim} соответственно. Данные характеристики Серелекса сравнимы с показателями существующих метрик семантической близости (см. ~\cite{panchenko2012konvens}), основанных на WordNet (\textit{WuPalmer}~\cite{wu1994verbs}, \textit{LeacockChodorow}~\cite{leacock1998}, \textit{Resnik}~\cite{resnik1995}), словарях (\textit{ExtendedLesk}~\cite{banerjee2003extended}, \textit{GlossVectors}~\cite{patwardhan2006using}, \textit{WiktionaryOverlap}~\cite{zesch2008extracting}) и  корпусах текстов (\textit{ContextWindow}~\cite{cruys2010mining}, \textit{SyntacticContext}~\cite{cruys2010mining}, ~\textit{LSA}~\cite{landauer1998introduction}).


\risunok{Результаты: задача ранжирования семантических отношений.}{pr}{8cm}


\subsubsection{Ранжирование семантических отношений}

В данном тесте нужно отсортировать некоторое множество слов по семантической близости с заданным словом. Например, дано 50 слов, 25 из которых семантически связаны со словом ``alligator'', в то время как 25 других с ним не связано. Задача заключается в ранжировании слов таким образом, чтобы семантически связанные пары имели более высокий ранг, например:

\begin{itemize}
\footnotesize
\item \texttt{1; alligator; animal (related)}
%\item \texttt{2; alligator; beast (related)}
\item \ldots
\item \texttt{25; alligator; lizard (related)}
\item \texttt{26; alligator; twin (random)}
\item \ldots
\item \texttt{50; alligator; electronic (random) }
\end{itemize}

Задача ранжирования основана на наборе семантических отношений BLESS~\cite{baroniwe} и SN~\cite{panchenko2012study}. В отличие от трех других тестов, данная задача позволяет оценить не только относительную точность, но и относительную полноту системы.

Точность Серелекса на данной задаче сопоставима с 9 указанными выше альтернативными метриками, однако полнота серьезно ниже в связи с разреженностью подхода, основанного на шаблонах (см. Рис \ref{fig-pr}). К примеру, \textit{SyntacticContext} достигает полноты 0.744, в то время как Серелекс достигает полноты около 0.389~\cite{panchenko2012konvens}. При оценке полноты следует также учитывать, что количество семантических отношений распределено экспоненциально~\cite{panchenko2013phd}. Поэтому большинство слов имеют только около 10-100 семантически связанных слов. 

\subsubsection{Извлечение семантических отношений} 

Кроме двух описанных выше тестов, была оценена точность извлечения семантических отношений для 49 слов из лексикона \textit{RG}. В данном эксперименте трем асессорам было предложено аннотировать результаты поиска и указать для каждого из 50 первых результатов, является ли он релевантным или нет. Например, для запроса ``fruit'':

\begin{itemize}
  \footnotesize
  \item \texttt{1; vegetable (relevant)}
  \item \texttt{2; mango (relevant)}
  \item \ldots
  \item \texttt{50; house (non-relevant)}
\end{itemize} 

На основании полученной статистики вычислена точность для $k$ первых результатов, где $k \in \{1, 5, 10, 20, 50\}$. Согласно результатам данного эксперимента, приведенным на Рис \ref{fig-eval2} (а), средняя точность извлечения варьируется между 74\% (для первого результата, $k=1$) и 56\% (50 первых результатов, $k=50$). Мы зафиксировали значительную степень согласия асессоров для данного эксперимента в терминах каппы Флейса: 0.61--0.80.

\risunok{Результаты: (a) задача извлечения семантических отношений; (б) удовлетворенность пользователей первыми 20 результатами поиска.}{eval2}{10cm}

\subsubsection{Удовлетворенность пользователей качеством поиска} 
Каждому из 23-х асессоров, участвующих в исследовании, было предложено выбрать 20 запросов по своему усмотрению и оценить первые 20 результатов поиска как релевантные, нерелевантные или как частично релевантные. В результате данной оценки было собрано 460 суждений асессоров и 233 суждения анонимных пользователей системы. Пользователи и асессоры вместе осуществили 594 уникальных запроса. В соответствии с этим экспериментом, результаты поиска являются релевантными для 70\% запросов и нерелевантными для 10\% запросов (см. Рис \ref{fig-eval2} (б)). В 20\% случаев первые 20 результатов оказались частично релевантными.

\subsection{Выводы}

Разработана система Серелекс, которая позволяет осуществлять поиск семантически связанных слов. Оценка качества работы системы на четырех экспериментах показала, что точность системы сопоставима с аналогичными существующими разработками. При этом, в отличие от большинства аналогов, Серелекс не использует составленные вручную словари. За счет этого достигается лучшее лексическое покрытие, так как семантические отношения извлекаются непосредственно из текста. Опрос пользователей показал, что первый результат поиска релевантен в 74\% случаях и в 70\% запросов пользователи полностью удовлетворены первыми 20 результатами.   

Мы работаем над построением аналогичной системы для французского и русского языков. При адаптации системы к новому языку мы планируем перевести набор шаблонов разработанных для английского языка. При этом будут использоваться стандартные словари и средства морфологического анализа, включенные в Unitex. Кроме того, мы работаем над интеграцией в систему модуля распознавания имен собственных (Named Entities Recognition), что позволит извлечь отношения не только между словами и словосочетаниями из словаря, но и между названиями компаний, именами публичных людей и т.п. 

\bibliographystyle{splncs}
\bibliography{biblio2}

\end{document}
